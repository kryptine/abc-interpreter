\documentclass[a4paper]{article}

\usepackage{geometry}

\usepackage[backend=biber,natbib]{biblatex}
\bibliography{library}

\title{Cross-Platform Dynamics in Clean}
\author{Camil Staps\footnote{Radboud University Nijmegen} \and Erin van der Veen\footnotemark[1]}

\begin{document}

\maketitle

\section{Clean}
\subsection{Graph Rewriting}
In conventional (imperative) programs the state of the program is defined by: the registers, the stack and the heap.
This is not the case in most modern functional programming languages.
In Clean, for example, the state is defined by: the registers, the A-Stack, the B-Stack, the C-Stack and the graph.
Graph rewriting (a more efficient version of term rewriting where the terms are represented as nodes in a graph) is used to reduce the program to it's normal form.
The way the graph must be rewritten is defined in the term rewriting rules.

When writing a Clean program one is, in essence, providing instructions on how the graph must be rewritten.

\subsection{ABC-code}
When Clean is compiled, the source code is first translated to ABC-code.
ABC-code has similarities to assembly instructions, with the difference that the ABC-machine is an abstract machine.
The fact that the ABC-Machine is an imperative machine ensures that the translation of a functional program to imperative machine code is a relatively small step.
This, in turn, makes it possible for Clean to target many different architectures.

\subsection{Soccer-Fun}
Soccer-Fun is an educational project written in Clean in which participants can write the AI for a simple football team.
Currently, there are three options if you want to play against an opponent:
\begin{enumerate}
	\item Send your source files to the other person
	\item Compile your Team of ABC-code and send that the other person
	\item Write dynamics to disk, and send those files.
\end{enumerate}
The first options is not ideal in an educational setting because it creates a setting where it is too easy for other students to copy (parts of) code from fellow students.
The second option is not ideal because it requires understanding of the compiler pipeline in Clean.
The third options is, unfortunately, impossible.
Dynamics cannot be shared between different binaries or architectures.

\subsection{Our Approach}
The idea of our research is to remove the requirement that dynamics can only be shared between instances of the same binary.
In order to do this, we need to not only serialize the current graph, but also the rewriting rules in form of ABC-code.
The ABC-code must then be interpreted by the other binary.

\printbibliography

\end{document}
