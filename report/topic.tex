\documentclass[a4paper]{article}

\usepackage[margin=15mm,bottom=2cm,top=2cm]{geometry}

\usepackage[backend=biber,natbib]{biblatex}
\bibliography{library}
\renewcommand*{\bibfont}{\small}

\title{Cross-Platform Dynamics in Clean}
\author{Camil Staps\footnote{Radboud University Nijmegen} \and Erin van der Veen\footnotemark[1]}

\begin{document}

\maketitle

\section*{Topic}
Dynamics are data types that store both a value and the type of that value.
Such types are useful in strongly typed languages to store values of different types in homogeneous data structures.
Clean is a pure, lazy, strongly-typed functional programming language with builtin support for dynamics.
On 32-bit Windows, it is possible to store a dynamic on disk, together with all the corresponding object code,
	which allows for sharing of dynamics between different applications.
This is also possible with expressions that are not completely evaluated.

Clean also supports serialization of any value, but these values can only be deserialized by the same executable.
Serialization can also be done using generic JSON functions, but these are not able to serialize unevaluated expressions.
Thus, there is currently no way to communicate values of arbitrary type between different executables on different platforms in a type-safe manner.

\section*{Research Approach}
In this study, we extend the Clean runtime system with a custom bytecode interpreter.
We add an executable-independent serialization function,
	which can look up the ABC code\footnote{ABC code is the intermediate target language of the Clean compiler. It is a high-level assembly language for an abstract graph rewriting machine.} corresponding to the serialized expression.
This ABC code is optimised and compiled to an efficient, platform-independent bytecode.
This bytecode can then be interpreted by any client application, allowing for type-safe cross-platform communication of arbitrary types.
If time permits, we can extend the system with a Just-in-Time compiler for better performance.

\section*{Related Work}
Dynamics have a long history~\cites[e.g.]{Plotkin1987,Abadi1992,Lero1993,Pil1998,VanNoort2012}.
The \emph{iTasks} system uses the intermediate functional language SAPL rather than ABC for client-side code execution~\cites{Jansen2006,Plasmeijer2008}.
We choose to use ABC because it is more efficient and provides for a shorter pipeline.
John van Groningen has already worked on an ABC optimiser, a code generator for 32-bit bytecode and a proof-of-concept interpreter.
This system needs to be extended with a garbage collector.
The bytecode needs to become platform-independent and the interpreter needs to be finished.
The system needs to be integrated with the dynamics system using new platform-independent serialization functionality.

\footnotesize
\printbibliography

\end{document}
