\documentclass{scrartcl}

\usepackage{authblk}

\title{Cross-Platform Dynamics in Clean}
\subtitle{Topic and Research Plan}

\author[1]{Camil Staps}
\author[1]{Erin van der Veen}
\affil[1]{Radboud University Nijmegen}

\begin{document}

\maketitle

\section{Topic}
Dynamics are data types that store both a value and the type of that value.
Such types are useful in strongly typed languages to store values of different types in homogeneous data structures.
Clean is a pure, lazy, strongly-typed functional programming language with builtin support for dynamics.
On 32-bit Windows, it is possible to store a dynamic on disk, together with all the corresponding object code,
	which allows for sharing of dynamics between different applications.
This is also possible with expressions that are not completely evaluated.

Clean also supports serialization of any value, but these values can only be deserializid by the same executable.
Cross-executable serialization can be obtained by generic JSON functions, but these are not able to serialize functions or unevaluated expressions.
Thus, there is currently no way to communicate values of arbitrary type between different executables on different platforms in a type-safe manner.

\section{Research Approach}
In our study, we extend the Clean runtime system with a custom bytecode interpreter.
We add an executable-independent serialization function,
	which can look up the ABC code\footnote{ABC code is the intermediate target language of the Clean compiler. It is a high-level assembly language for an abstract graph rewriting machine.} corresponding to the serialized expression.
This ABC code is optimised and compiled to an efficient, platform-independent bytecode.
This bytecode can then be interpreted by any client application, allowing for type-safe cross-platform communication of arbitrary types.

\end{document}
