\documentclass{scrartcl}

\usepackage{authblk}

\title{Cross-platform Dynamics in Clean}
\subtitle{Topic and Research Plan}

\author[1]{Camil Staps}
\author[1]{Erin van der Veen}
\affil[1]{Radboud University Nijmegen}

\begin{document}

\maketitle

\section{Topic}
Dynamics are data types that store both a value, and a descriptor of the type of that value.
Such types allow storing different types in data structures that are typically homogeneous.
Clean is a pure lazy functional programming language that has an implementation of dynamics.
In Clean, dynamics are implemented as a language construct.
The fact that they are implemented as such has the added benefit that it is possible to serialize them.
In fact, it is even possible to serialize dynamics that have not been completely evaluated.

Currently, Clean has the ability to serialize the dynamics (by using a string representation of the internal graph), but the dynamic can subsequently only be deserialized by the executable that serialized it.
The reason for this limitation is that the graph rewriting code is only available to this executable (including relocations).

\section{Research Approach}
In our research, we extend the Clean runtime system with a ABC-interpreter\footnote{ABC-code is the code that describes how a graph should be rewritten} and an executable independent \texttt{graph\_to\_string} function.

\end{document}

